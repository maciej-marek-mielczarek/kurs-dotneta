\documentclass[10pt]{article}

\usepackage[utf8]{inputenc} % linux
%\usepackage[cp1250]{inputenc} % windows
\usepackage[T1]{fontenc}
\usepackage{hyperref}
\usepackage{polski}
\usepackage[polish]{babel}

\setlength{\parskip}{1em}

\title{Notatki do 1. tygodnia kursu Dotneta}
\author{Maciek Mielczarek}

\begin{document}

\maketitle


\section{Co ja czytam?}

W tej serii dokumentów zamierzam opisywać drobne problemy pojawiające się podczas przerabiania kursu, w miarę możliwości razem z rozwiązaniami. Będą tu też opisy tego co robię inaczej na Linuxie niż na Windowsie. Staram się nie zakładać, że świetnie znasz język angielski.


\section{Instalacja}

Problem:
\begin{itemize}
\item Mam już zainstalowane Visual Studio Community i okazało się że to wersja demonstracyjna, tylko na 30 dni.
\end{itemize}
Rozwiązanie:
\begin{itemize}
\item Zaloguj się do swojego konta Microsoft w Visual Studiu, aby zmienić licencję. Odpowiednie miejsce do logowania to $"$Help -> About Microsoft Visual Studio -> License status (w górnym prawym rogu nowego okna)$"$. Jeśli nie masz konta Microsoft, to je stwórz.
\end{itemize}


Problem:
\begin{itemize}
\item Chcę przerabiać ten kurs na Linuxie (albo Macu), ale nie ma Visual Studia na Linuxa (ani Maca).
\end{itemize}
Rozwiązanie:
\begin{itemize}
\item Rzeczywiście, nie ma. Są jednak inne IDE, jak VS Code i Monodevelop. Mój główny system to Linux i na nim również (mam też dostęp do Windowsa) będę się starał wykonywać zadania z kursu i opisywać związane z tym problemy. Jeśli więc masz zamiar przerabiać kurs na Linuxie, to wiedz, że nie jesteś z tym sam(a). O Macach się nie wypowiadam, bo nie mam żadnego pod ręką żeby cokolwiek na nim sprawdzać.
\end{itemize}

Problem:
\begin{itemize}
\item To co w końcu mam zainstalować na Linuxie?
\end{itemize}
Rozwiązanie:
\begin{itemize}
\item VS Code,
\item Monodevelop,
\item Azure Data Studio (chyba mniejwięcej odpowiednik SQL Server Management Studio),
\item Postman.
\end{itemize}

\section{Hello World}

\subsection{...bez IDE}

Jak stworzyć i uruchomić swój pierwszy program w C\# na Linuxie:
\begin{enumerate}
\item Zainstaluj $"$.NET Core SDK$"$
\item Sprawdź, czy jest zainstalowane: Wpisz w terminalu $"$dotnet$"$.
\item Jeśli wyświetlą się jakieś informacje o dotnecie, to kontynuuj, w przeciwnym przypadku wróć do kroku 1.
\item Wpisz $"$dotnet new console -o myApp$"$.\\
To stworzy aplikację konsolową $"$Hello World$"$ w folderze myApp.
\item cd myApp\\
dotnet run
\end{enumerate}
Trochę cierpliwości. To wprawdzie tylko $"$Hello World$"$, ale i tak uruchamianie chwilę trwa.

Jak stworzyć więcej programów konsolowych:
\begin{enumerate}
\item dotnet new console -o nazwaProgramu
\item Edytuj plik ./nazwaProgramu/Program.cs
\end{enumerate}

Jak je uruchomić:
\begin{enumerate}
\item Przejdź do folderu $"$nazwaProgramu$"$
\item dotnet run
\end{enumerate}

\subsection{...w Monodevelopie}

\begin{enumerate}
\item Wejdź w File -> New Solution.
\item Wybierz $"$App$"$ pod $"$.NET Core$"$.
\item Wybierz $"$Console Application$"$ i kliknij next.
\item Odznacz $"$Create a project directory within the solution directory$"$, aby uzyskać strukturę folderów taką jak na kursie. Kliknij create.
\item Wybierz z menu Run -> Run without debugging.
\end{enumerate}

Jeśli program nie zadziała (to jest nie zobaczysz okienka z napisem $"$Hello World!$"$) i zobaczysz wiadomość podobną do tej:
\begin{quote}
It was not possible to find any compatible framework version\\
The framework 'Microsoft.NETCore.App', version '1.1.2' was not found.\\
\hspace*{2em}- The following frameworks were found:\\
\hspace*{6em}3.1.6 at [/usr/share/dotnet/shared/Microsoft.NETCore.App],
\end{quote}
to kliknij File -> Close Solution, następnie otwórz plik /HelloWorld/HelloWorld.csproj i w linii:\\
<TargetFramework>netcoreapp1.1</TargetFramework>\\
zmień numer $"$1.1$"$ (bo w komunikacie o błędzie jest $"$wersja 1.1.coś nie została znaleziona$"$) na $"$3.1$"$ (bo w komunikacie zaraz potem jest $"$znaleziono (...) 3.1.coś$"$). Potem każ Monodevelopowi otworzyć zamknięte przed chwilą rozwiązanie i ponownie wykonaj ostatni krok (Run -> Run without debugging). Teraz powinno zadziałać.

\section{Praca z IDE}

W Monodevelopie tworzenie nowych, pustych klas przebiega tak jak w VS, natomiast w konsoli wśród komend $"$dotnet new...$"$ nie ma takiej tworzącej pustą klasę. Stworzenie odpowiedniego szablonu przypisującego  nowemu plikowi i klasie nazwę najwyraźniej nie jest trywialne. Jeśli będę tworzył dużo nowych, pustych klas, to może napiszę do tego jakiś skrypt w bashu.

Breakpointy w Monodevelopie nie działają i nie udało mi się ich uruchomić. Gdy będę chciał poznać wartości jakichś zmiennych w jakimś miejscu programu, to zawsze mogę je wprost wypisać.

Wracam do VS. Nie widziałem narzędzi diagnostycznych, ale kliknięcie w Debug -> Windows -> Show Diagnostic Tools podczas działania programu włącza je.

\section{Kompilowanie}

Ciekawe jak to wygląda z kompatybilnością? Sprawdzam... Pliki wykonywalne przenoszone między Windowsem a Linuxem oczywiście nie działają i Wine automagicznie tego nie zmienia. Projekty kopiowane między systemami otwierają i uruchamiają się natomiast bez problemów.

\section{Debugowanie}

Zauważyłem, że podczas debugowania widoczne są zmienne, które zostaną stworzone dopiero za kilka linijek. Sprawdziłem, że schowanie tych kolejnych linijek w nawiasie klamrowym chowa te zmienne przed debuggerem do momentu wejścia do środka tego nawiasu.

\end{document}
