\documentclass[10pt]{article}

\usepackage[utf8]{inputenc} % linux
%\usepackage[cp1250]{inputenc} % windows
\usepackage[T1]{fontenc}
\usepackage{hyperref}
\usepackage{polski}
\usepackage[polish]{babel}

\setlength{\parskip}{1em}

\title{Notatki do 2. tygodnia kursu Dotneta}
\author{Maciek Mielczarek}

\begin{document}

\maketitle

\section{Wstęp}
Na ten tydzień zaplanowane są podstawy języka C\#. Najprawdopodobniej będę porównywał wprowadzane elementy do tego co znam z C++ albo Javy.

\section{Wybór projektu}
Nie wiem co wybrać, więc rzucę kostką. Na początek właściwe losowanie przy użyciu fizycznych kości k6 (6 ścian z numerami od 1 do 6). Są różne możliwości wylosowania liczby od 1 do 25 przy pomocy standardowych kostek. Zdecydowałem się na następujący wariant:
\begin{enumerate}
\item Z k6 robię k5 przez rzucenie ponownie w przypadku wyniku 6. Dzięki temu wszystkie powtórki rzutów załatwiam od razu.
\item 2 razy rzucam k5 i odejmuję od wyniku 1, aby dostać cyfrę dziesiątek (a właściwie piątek) i cyfrę jedności liczby w systemie piątkowym.
\item Przeliczam tą liczbę do systemu dziesiętnego (pierwsza cyfra razy 5 plus druga cyfra) i dodaję 1, żeby dostać liczbę z przedziału od 1 do 25.
\end{enumerate}

Pierwszy rzut: 5. Drugi rzut: 3. To odpowiada liczbie 23, czyli grze w kółko i krzyżyk. To prawdopodobnie najprostszy temat, ale ponieważ istnieje wiele oczywistych wariantów tej gry, to będę mógł sprawdzić czy zaplanowałem i napisałem kod w taki sposób, żeby aplikację dało się rozwijać.

Skoro już jestem przy temacie rzucania kostką, to sprawdzę jak się losuje liczby w C\# i odtworzę powyższą sytuację w kodzie. Pewnie wiele rzeczy jest nie na swoim miejscu lub zrobionych dziwnie, ale kod jest przetestowany i działa. Można go znaleźć \href{https://github.com/maciej-marek-mielczarek/kurs-dotneta/tydz\_02/Dices}{tutaj}.
\end{document}