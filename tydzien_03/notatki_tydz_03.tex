\documentclass[10pt]{article}

\usepackage[utf8]{inputenc} % linux
%\usepackage[cp1250]{inputenc} % windows
\usepackage[T1]{fontenc}
\usepackage{hyperref}
\usepackage{polski}
\usepackage[polish]{babel}

\setlength{\parskip}{1em}

\title{Notatki do 3. tygodnia kursu Dotneta}
\author{Maciek Mielczarek}

\begin{document}

\maketitle

\section{Wstęp. Stan aplikacji po 2. tygodniu}
U mnie jak i u przynajmniej kilkorga innych kursantów aplikacja ma przynajmniej 1 plik który ma kilkaset linii. Prawdopodobnie można to uznać za wyznacznik bałaganu, potrzeby refactoru i powtarzającego się kodu. Aplikacje jednak w jakimś stopniu działają.

\section{Konstruktory}
Stworzenie jakiegokolwiek konstruktora powoduje nie tworzenie konstruktora domyślnego.

Słowo this jest do wołania innych konstruktorów tej samej klasy. Wpisujemy je po dwukropku, jak wywołanie funkcji z parametrami, między listą parametrów a ciałem konstruktora. W tym miejscu możemy zamiast this użyć base, by wywołać konkretny konstruktor klasy bazowej.

Można zainicjalizować pola zaraz za wywołaniem konstruktora domyślnego, w nawiasach klamrowych.

\section{Przeciążanie metod}

\end{document}
