\documentclass[10pt]{article}

\usepackage[utf8]{inputenc} % linux
%\usepackage[cp1250]{inputenc} % windows
\usepackage[T1]{fontenc}
\usepackage{hyperref}
\usepackage{polski}
\usepackage[polish]{babel}

\setlength{\parskip}{1em}

\title{Notatki do 4. tygodnia kursu Dotneta}
\author{Maciek Mielczarek}

\begin{document}

\maketitle

\tableofcontents

\section{Projekt testowy}
Dla C\# mamy 3 popularne biblioteki do testowania aplikacji, będące bazami dla 3 typów projektów testowych:
\begin{itemize}
\item MSTest najstarszy, niemrawo rozwijany przez Microsoft,
\item NUnit, xunit tworzone przez społeczność dotnetową,
\item xUnit ma z tych 3 najprostszą składnię testów, więc to dobry wybór na początek.
\end{itemize}

Pojawiają się anotacje - słowa w nawiasach kwadratowych tuż nad metodami. Anotacja $"$Fact$"$ mówi, że zaznaczona metoda to metodą testową.

Wciśnięcie PPM na projekcie testowym w eksploratorze rozwiązania -> run tests, jak nazwa wskazuje, odpala testy.

Luźno latające okienka w IDE (przynajmniej niektóre), takie jak to z testami można, przez przeciągnięcie gdzieś do rogu, $"$zadokować$"$, czyli sprawić, żeby zostały w wyznaczonym miejscu i nie znikały po użyciu.

Zielone i czerwone kółka przy metodach testowych oznaczają wynik ostatniego testu i są skrótami do odpalania tego testu.

\end{document}
